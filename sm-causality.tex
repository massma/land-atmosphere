\documentclass[12pt]{article}
\usepackage{float}
\usepackage{wrapfig}
\input{def}
\graphicspath{{figs/}}

\begin{document}

\title{Estimating the ecosystem evaporation response to interventions
  on soil moisture: confounding and causal modeling in a simulated
  world}

\author{Adam Massmann\thanks{Corresponding author:
    akm2203@columbia.edu}, Pierre Gentine, Olya Skulovich}

\maketitle

\section{Introduction}

In Chapter \ref{cha:when-does-vapor} we developed theory for the
evaporation response to changes in atmospheric aridity, and we
encountered qualitative differences in current plant models'
representation of the ecosystem response. This divergence in plant
models' behavior is a reflection of the challenges of modeling such
complex systems with necessarily simple models with relatively few
parameters. However, it also motivates the exploration of other
methods of developing plant models and estimating plant response to
interventions on environmental conditions.

One alternative method for developing plant models is to more strongly
guide their response with observational data from the real world. We
know such data are representative of ecosystems' full complexity and
relationship with the environment, so perhaps we can use those data to
develop models or computational methods that are faithful to
ecosystems' real world behavior. In order to replace more traditional
theoretical models, any models derived from data must estimate the
response to interventions on initial and/or boundary conditions,
rather than just observed correlations between responses and
initial/boundary conditions. This distinction and need for models
based on interventions rather than correlations is crucial: as we saw
in Chapter \ref{cha:caus-infer-proc}, naive correlation-based
relationships between two variables can be attributable to other
processes entirely; for example, a correlation relationship between
soil moisture and evaporation could be partially attributable to
synoptic-induced co-variations between soil moisture, vapor pressure
deficit, cloudiness, etc., all of which affect evaporation. If we used
such a correlation-based relationship in a numerical experiment on
soil moisture, or alternatively within a larger model that supplies a
soil moisture boundary condition, then the calculated response of
evaporation would be biased. However, if we are able to use data to
isolate the interventional and causal relationship between soil
moisture and evaporation, then we can use such an interventional
relationship within numerical experiments and larger system-based
models without a risk of confounding bias (see Chapter
\ref{cha:caus-infer-proc} for more on confounding bias). Developing
methods to extract interventional relationships from passive
(i.e. non-experimental) observations is the primary goal of this
chapter.

This data-driven approach and the need to calculate the response to
interventions introduces challenges: as we saw in Chapter
\ref{cha:caus-infer-proc}, differentiating causal
(i.e. interventional) relationships from mere correlations requires
controlling for sources of confounding\cite{pearl2009}, which can be
difficult particularly in a system with feedbacks and an underlying
climate state that drives co-variations in many of the processes of
interest. Given these potential challenges, an intermediate step
before calculating a causal effect with real world data is to estimate
a causal effect from passively observed data in a simulated reality
where we know the true response to interventions. If our estimated
causal effects match the simulation's true response to interventions,
and the simulated reality shares many of the characteristics that make
causal inference challenging in the real world, then we have
confidence about our method's potential success in the real world.

However, if we had access to a simulation that reproduces reality,
then we would have no need to estimate causal relationships from data.
Unfortunately any simulation will deviate from reality. However, we do
not need a simulator that matches reality; we just need a simulator
with a reasonable amount of realism, and a simulator that shares the
same methodological challenges to causal inference and estimation, for
example feedbacks, non-linearity, and perhaps most importantly,
co-variations in the initial and boundary conditions due to the
underlying climate and land surface state (for example: soil moisture,
vapor pressure deficit, and cloudiness). We will use a simulator with
these characteristics to accomplish our two primary goals:

\begin{enumerate}
  \item Quantify the degree to which confounding shifts the
    relationship between simulated evaporation and soil moisture, and
    how confounding varies across site and climate.
\item Develop methods for calculating a causal effect from data that
  are tractable and pragmatic in the real world. We will apply causal
  estimation to simulated data where we know the true response of
  evaporation to interventions on soil moisture, and assess the
  reliability of our causal estimates relative to the simulated truth.
\end{enumerate}

\section{Methods}
\label{sec:methods-sm-caus}

This research project has two goals: 1) estimate and present the
impact of confounding on observationally derived relationships between
soil moisture and ecosystem evaporation, and 2) develop a method for
calculating causal relationships between interventions on soil
moisture and ecosystem evaporation, and assess the method's
performance and biases in a simulated world where we know the true
effect.

To accomplish both of these goals, we will use the Chemistry
Land-surface Atmosphere Soil Slab (CLASS) model
\cite[\url{https://classmodel.github.io/}, ]{de2015atmospheric,},
which was developed specifically for representing and understanding
land-atmosphere feedbacks. Furthermore, there is a database of initial
and boundary conditions for CLASS called CLASS4GL
\cite[\url{https://class4gl.eu/}]{wouters-class4gl-2019}. CLASS4GL
assimilates weather balloon, satellite, and reanalysis data to
generate initial and boundary conditions to the CLASS model that are
consistent with the observed state at sites around the world. This
database was also designed specifically to study land atmosphere
feedbacks, and most importantly for our applications, it will include
the true observed co-variability between initial conditions. So while
our simulated reality alters the mapping between initial conditions
and the outcome of interest (evaporation) relative to the real world,
the co-variability of initial and boundary conditions matches the
truth in the real world. Given that the source of confounding between
soil moisture and evaporation is this climate-driven co-variability of
initial/boundary conditions, CLASS4GL's representation of this
climate-driven co-variability improves our confidence that methods
developed in CLASS's simulated reality may translate to the real
world.

To summarize, the simulated model (CLASS) is ideally suited to our
purposes: it is efficient enough to simulate many interventional
experiments for validation of our causal methods, and it is explicitly
designed to represent the feedbacks and non-linearity of the
system. Through CLASS4GL, we also have access to initial and boundary
conditions that are consistent with observed conditions, and represent
confounding due to the climate, synoptic, and land surface state.

We use every CLASS4GL site that has at least 500 daily initial and
boundary conditions, leaving 12 sites. This range of sites spans
various climates and plant functional types, which allows us to
understand confounding and our ability to adjust for it in terms of
climate and ecosystem type.

\subsection{Estimating the impact of confounding on observationally derived relationships between soil moisture and ecosystem evaporation}
\label{sec:estim-impact-conf}

To estimate the impact of confounding on observationally-derived
relationships between soil moisture and ecosystem evaporation, we will
compare ordinary linear least-square regressions from two simulated
worlds:

\begin{enumerate}
  \item reality; where soil moisture is confounding by the large scale
    climate and land surface state. We will call this the
    ``\textbf{Realistic World}.''

  \item an alternate reality where soil moisture is sampled
    independently of all other variables, and there is no
    confounding. We will call this the ``\textbf{Deconfounded
      World}.''
\end{enumerate}

\begin{figure}
  \includegraphics[width=\textwidth]{./sm-causality-figs/simulations.pdf}
  \caption{A diagram of this analysis's simulated worlds and the
    statistical models fit to those
    simulations.}\label{fig:simulations}
\end{figure}

For each world, we simulate the daily evolution of the boundary layer
for each set of daily initial and boundary conditions in the CLASS4GL
database. The only difference between the Realistic World and the
Deconfounded World is that in the Deconfounded World we use a soil
moisture initial condition that is randomly sampled independently from
the model's other initial and boundary conditions (Figure
\ref{fig:simulations}).

We can estimate the true causal effect for each world and each day by
varying the initial condition of soil moisture by $\pm0.01$. Our
outcome of interest is average evaporation from 1300 local time to
1700 local time. We can then use these outcomes from our local
interventions on soil moisture ($E(SM-0.01), E(SM), E(SM+0.01)$, where
$E$ is the average evaporation and $SM$ is initial soil moisture) to
numerically estimate the slope of the evaporation response to soil
moisture ($\frac{\partial E}{\partial SM}$) for each day. However,
occasionally non-linearity in the simulation leads to sets of three
points that are not in an approximately straight line, and it is not
clear what a single value of the ``true'' slope should be (Figure
\ref{fig:true-fit}). We represent this ambiguity by using three
different numerical methods to calculate the slope: the forward
difference method ($\frac{\partial E}{\partial SM} = \frac{E(SM+0.01)
  - E(SM)}{0.01}$), the backward difference method ($\frac{\partial
  E}{\partial SM} = \frac{E(SM) - E(SM-0.01)}{0.01}$), and the
centered difference method ($\frac{\partial E}{\partial SM} =
\frac{E(SM+0.01) - E(SM-0.01)}{0.02}$). We interpret these three
estimates as three possible values of the slope $E(SM)$ with respect
to $SM$, and we can use these three possible values to bound our
statistics. For example, we will compare data-driven causal estimates
of the average evaporation response to soil moisture to averages of
our ``true'' slope. We calculate the range of expectation of the true
slope by taking the average of the minimum of the forward- and
backward- difference estimates of the slope for each day, and the
average of the maximum of the forward- and backward- difference
estimates of the slope for each day. In this way, we can represent the
range of possible expectations of our ``true'' slope with the averages
of the minimums and maximums of the slopes at each day.

\begin{figure}
  \includegraphics[]{./sm-causality-figs/true-fit.pdf}
  \caption{Estimates of the ``True'' causal response of evaporation to
    soil moisture by varying soil moisture by $\pm0.01$ with all other
    initial and boundary conditions held fixed. For most days, like
    Julian day 242 of year 1998, the response is approximately linear
    and there is a well defined slope (top panel). However, for some
    days, like Julian day 256 of year 1998 (bottom panel), the three
    points are visibly non-linear. Because of this ambiguity, we
    calculate the slope three ways: using the centered difference,
    forward difference, and backward difference.}\label{fig:true-fit}
\end{figure}

We will compare the expectation of the ``true'' response to soil
moisture, to a naive approach of calculating the evaporation response
of soil moisture, where we just take the linear least squares
regression of daily average evaporation between 1300 and 1700 local
time onto soil moisture's initial condition. In the Realistic World,
the bias between the naive regression approach and the ``true''
response to soil moisture includes model specification bias (for
example, using a linear least squares regression for a non-linear
process) and confounding bias (for example, co-variations in soil
moisture and evaporation that are due to the underlying climate
instead of the causal effect of soil moisture on evaporation).

We can compare the naive approach's bias in the Realistic World to
bias of the same naive regression approach in the Deconfounded
World. In the Deconfounded World, there is no systematic confounding
error because soil moisture is sampled independently from the other
initial and boundary conditions of the model, so in the Deconfounded
World the bias of the naive approach is attributable to model
specification bias. In order to understand the magnitude of
confounding bias, and how it varies by site, we will compare the error
of the naive approach in the Realistic World (model specification +
confounding error) to the error of the naive approach in the
Deconfounded World (only model specification error).

We use a linear model because of its interpretability, simplicity, and
freedom from hyper-parameters that could be more or less suited to one
of the two simulated worlds. We also use a linear model because it
represents in some ways a ``worst case'' of model specification error
for a nonlinear process like evaporation. Comparing the confounding
bias to model specification bias in an extremely naive and simple
linear model should help contextualize the magnitude of confounding
bias: due to their ubiquitous use, linear models' errors are
relatively well understood for these processes, so a comparison
relative to these errors should be instructive. It is also the case
that no model specification will fit exactly the same in both the
Deconfounded and Realistic Worlds, but because of physical science's
history with using linear models and assessing their
fit\cite{shalizi2013}, at least we have some well established
procedures and intuition for assessing the degree to which linear
models fit the data (for example, examining patterns in residuals to
assess the degree to which they violate the assumptions of ordinary
least squares linear regression).

\subsection{Adjust for confounding: a simple approach and its errors}
\label{sec:attempt-meth-adjust}

To adjust for confounding, we fit a simple, interpretable statistical
model to simulated data from the Realistic World, and compare the
statistical model's performance to the estimation of the ``true''
causal effect.

To account for the sources of confounding, we do a nearest-neighbor
search for nearby points, where nearness is defined as the euclidean
distance of normalized data for soil moisture and all of the
confounders (leaf area index, ground temperature, air temperature,
humidity, cloud cover, wind-speed, boundary layer height [measure of
  wind and stability], day of year [driver of the top of the
  atmosphere incoming radiation], Figure \ref{fig:graph}).  For each
day, we take the k-nearest neighbors, and for those neighbors we
regress evaporation onto soil moisture to estimate the local (to that
day) response of evaporation to soil moisture. Because of this linear
regression approach on the nearest neighbors, we include soil moisture
in the measure of nearness to account for nonlinearity: soil
moisture's slope will vary as a function of soil moisture, and
including soil moisture in the measure of nearness is a pragmatic
method to allow for that nonlinearity in the model. The number of
nearest neighbors ($k$) is determined by varying $k$ and selecting the
$k$ that results in the lowest mean squared error of the regression on
held out data.

\begin{figure}
  \includegraphics[width=\textwidth]{./sm-causality-figs/graph.pdf}
  \caption{A causal graph of the land atmosphere system in the CLASS
    model. The causal effect of soil moisture on evaporation is
    confounded by the season and climate state. However, if we control
    for leaf area index (LAI), temperature, humidity, cloud cover,
    wind-speed, boundary layer height, and day of year, the causal
    effect of soil moisture on evaporation is identified. }
  \label{fig:graph}
\end{figure}

We use the k-nearest neighbors approach because:

\begin{itemize}

\item it allows for nonlinearities in the confounders.
\item it bears a conceptual similarity to the adjustment technique of
  matching \cite{stuart2010matching}, where one searches for samples
  with the same (or similar) observations of the confounders, and
  compares the outcome (in this case evaporation) to the cause (in
  this case soil moisture) \cite{shalizi2013}.
\item both the nearest neighbor search and the linear regression steps
  are transparent and interpretable relative to more black box
  regression techniques.
\item the algorithm is deterministic, which aides interpretability and
  saves us from needing to bootstrap random sampling of initial values
  of parameters.
\item it relies on only one hyper-parameter, $k$
\item it allows for local estimates of causal effects (e.g., at a
  daily timescale) in addition to longer-term averages. However these
  local estimates of causal effects may be quite noisy, and for this
  manuscript we focus on analyzing site-level averages of the soil
  moisture response to aridity.
\end{itemize}


A summary of the steps of the k-nearest neighbors estimation method is
as follows:

\begin{enumerate}
\item \textbf{Aggregate and normalize the data}: Aggregate daily
  initial (dawn) measurements of the confounders (leaf area index,
  ground temperature, air temperature, humidity, cloud cover,
  wind-speed, boundary layer height [measure of wind and stability]),
  initial (dawn) measurements of soil moisture (the cause), and
  average afternoon evaporation (the effect, averaged from between
  1300 and 1700 local time). Subtract the mean and normalize each
  variable by its standard deviation to make the variables more
  equally weighted in the Euclidean distance metric used to determine
  the $k$ nearest neighbors, where all of the confounders and soil
  moisture are the dimensions for the distance.
  \item \textbf{Determine the hyperparameter $k$}: Select the
    hyperparameter $k$ that minimizes the squared error on held out
    data of the local regression, to avoid over-fitting and optimize
    the predictive power of our regression. For each $k$ in $\{5, 10,
    20, 30, 40, 50, 60, 70, 80, 90, 100\}$, evaluate the average
    squared error of a local linear regressions on held out data:
    \begin{enumerate}
    \item{Find the $k$ nearest neighbors for each day. Randomly split
      the $k$ nearest neighbors into a 80\%/20\% train/test split, and
      fit a linear regression model on the training set. Then evaluate
      the squared error of the linear model on the testing
      set. Bootstrap and resample the train/test split 5 times, and
      average the squared error over each train/test
      resampling.}
    \item{Average the squared error over each day.}
    \end{enumerate}
    Lastly, compare the average squared error on held out data as a function
    of $k$, and select the $k$ that minimizes the average squared error.
  \item \textbf{Calculate an evaporation response slope for each day:}
    Find the $k$ nearest neighbors, where $k$ is determined in the
    above previous step. For the $k$ nearest neighbors, fit a linear
    model between initial (dawn) soil moisture and average afternoon
    evaporation (between 1300-1700). The slope of the linear
    regression is used as the response of evaporation to a causal
    intervention on soil moisture ($\frac{\partial E}{\partial SM}$).
  \item \textbf{Calculate an average causal slope of evaporation
    response from the daily slopes:} Following
    Chapter \ref{cha:caus-infer-proc}, we approximate the average
    effect of changing soil moisture as the average of each of the
    daily slopes of the regression of afternoon evaporation onto soil
    moisture, as calculated in the previous step.
\end{enumerate}


\subsection{Site and their characteristics}

We use all sites in the CLASS4GL database with over 500 (daily)
observations (Figure \ref{fig:map}), leaving 12 sites: Bergen, Idar
Oberstein, Lindenberg, Milano, Kelowna, Quad City, Spokane, Flagstaff,
Elko, Las Vegas, Riverton, and Great Falls. Roughly these sites can be
grouped into two types: more humid/less water limited (Bergen, Idar
Oberstein, Lindenberg, Milano, Kelowna, and Quad City), and more
arid/water limited (Spokane, Flagstaff, Elko, Las Vegas, Riverton and
Great Falls). However, within those groups there is a broader range of
variability (Figure \ref{fig:site-climate}, and the following
paragraphs provide context on each site's individual characteristics
and location.

\begin{figure}
  \includegraphics[]{./sm-causality-figs/map.pdf}
  \caption{A map of all sites used in the analysis. These include all
    of the sites of the CLASS4GL database \cite{wouters-class4gl-2019}
    which had more than 500 daily observations.  }
 \label{fig:map}
\end{figure}


\begin{figure}
  \includegraphics[width=\textwidth]{./sm-causality-figs/site-climate.pdf}
  \caption{Violin plots of CLASS4GL observations of soil moisture (top
    panel), relative humidity (middle panel), and cloud cover (bottom
    panel). Observations are at or near dawn, and serve as
    initial conditions to the CLASS simulations used in this
    analysis. The wilting point, which is the threshold of soil
    moisture below which ecosystems no longer evaporate water, and the
    field capacity, which is the threshold of soil moisture above
    which evaporation is no longer reduced or impacted by soil
    moisture are shown for reference in the top panel.}\label{fig:site-climate}
\end{figure}

\paragraph{Bergen (Germany; latitude: 52.8153, longitude 9.9247,
  elevation: 70 m)} Bergen is in northern Germany about halfway
between Hanover and Hamburg. It has a temperate oceanic climate with
no dry season \cite[K\"{o}ppen: Cfb,][]{rubel2010}). It has slightly
elevated relative humidity relative to the other humid sites (Figure
\ref{fig:site-climate}).

\paragraph{Idar Oberstein (Germany; latitude: 49.6928, longitude:
  7.3264, elevation: 376 m)} Idar Oberstein is in western Germany,
about 100 km to the southwest of Frankfurt. It has a temperate oceanic
climate with no dry season \cite[K\"{o}ppen: Cfb,][]{rubel2010}, and
aridity characteristics that are relatively similar to the other humid
sites (Figure \ref{fig:site-climate}).

\paragraph{Lindenberg (Germany; latitude: 52.2167, longitude:
  14.1167, elevation: 112 m)}

Lindenberg is about 50 km to the southeast of Berlin, Germany. It has
a temperate oceanic climate with no dry season \cite[(K\"{o}ppen:
  Cfb)][]{rubel2010}, and aridity characteristics that are relatively
similar to the other humid sites (Figure \ref{fig:site-climate}).

\paragraph{Milano (Italy; latitude: 45.4614, longitude: 9.2831, elevation: 104 m)}

Milano is a major city in northern Italy. It has a humid subtropical
climate (K\"{o}ppen: Cfa) with no dry season. There are areas of
temperature ocean climate K\"{o}ppen: Cfb in the vicinity to the north
\cite{rubel2010}. The ecosystem is not very water limited: soil
moisture is above field capacity, which is the threshold of soil
moisture above which evaporation is no longer reduced or impacted by
soil moisture. For about 8\% of days in the CLASS4GL data. However,
relative to the other humid sites, Milano's climate is characterized
by lower cloud cover (Figure \ref{fig:site-climate}), and a broad
range of soil moisture values, suggestive of a quasi-Mediterranean
influence decreasing soil moisture in the summer.

\paragraph{Kelowna (Canada; latitude: 49.9408, longitude: -119.4003,
  elevation: 456 m)}

Kelowna is located in the interior of south-eastern Canada, about
300 km northeast of Vancouver. Kelowna has a subarctic climate
(K\"{o}ppen: Dfc), with warm-summer humid continental climates in the
vicinity to the northeast (K\"{o}ppen: Dfb)
\cite[][]{rubel2010}. Relative to the other humid sites, there are
slightly more clouds at Kelowna and also lower initial relative
humidity (Figure \ref{fig:site-climate}).

\paragraph{Quad City (United States; latitude: 41.6114, longitude:
  -90.5817, elevation: 230 m)}

Quad City straddles the Mississippi river on the border of Illinois
and Iowa, about 230 km west of Chicago in the United States. It has a
hot summer humid continental climate \cite[K\"{o}ppen:
  Dfa][]{rubel2010} with no dry season. Quad city is slightly more
humid and has slightly elevated soil moisture, but has fewer clouds
than the majority of the other humid sites (Figure \ref{fig:site-climate}).

\paragraph{Spokane (United States; latitude: 47.6806, longitude:
  -117.6267, elevation: 729 m)}

Spokane is located in the interior of the northwestern United
States. It has a warm-summer Mediterranean climate (Csb), with a
temperate oceanic climate (K\"{o}ppen: Cfb) in the vicinity to the NE
\cite{rubel2010}. This site is slightly arid during the growing
season: soil moisture is below the wilting point, defined as the
point at which the ecosystem can no longer sustain evaporation, for
about 4\% of days in the CLASS4GL data. It is among the more humid of
the arid sites, although it is characterized by low cloud cover
(Figure \ref{fig:site-climate}).

\paragraph{Flagstaff (United States; latitude: 35.23, longitude: -111.8217, elevation: 2179 m}

Flagstaff is located in the southeastern United States, on the
southwestern side of the Colorado Plateau. It has a warm-summer
Mediterranean climate \cite[K\"{o}ppen: Csb,][]{rubel2010}. Due to
complex terrain, there are also cold semi-arid climates
\cite[K\"{o}ppen: BSk,][]{rubel2010} to the southwest, and hot-summer
Mediterranean climates to the southeast \cite[K\"{o}ppen:
  Csa,][]{rubel2010}. This site has growing season aridity: about 16\%
of days are below the wiling point in the CLASS4GL data. Flagstaff is
among the more humid of the arid sites (Figure \ref{fig:site-climate}).

\paragraph{Elko (United States; latitude: 40.86, longitude: -115.7422,
  elevation: 1593 m)}

Elko is located in the western interior of the United States, about
500 km north of Las Vegas. It has a Mediterranean-influenced
warm-summer humid continental climate (K\"{o}ppen: Dsb), with
non-Mediterranean warm-summer humid continental climates (K\"{o}ppen:
Dfb) in the vicinity to the northeast \cite{rubel2010}. This site is
arid: about 30\% of days are below the wilting point in the CLASS4GL
data. It is second only to Las Vegas in terms of overall site aridity
(Figure \ref{fig:site-climate}).

\paragraph{Las Vegas (United States; latitude: 36.05, longitude:
  -115.1833, elevation: 697 m)}

Las Vegas is located int he southeastern U.S., and has a subtropical
cold desert climate (K\"{o}ppen: BWk), with hot desert climate in the
vicinity to the SE (K\"{o}ppen: BWh) \cite{rubel2010}. Fifty-five
percent of days are below the wilting point in the CLASS4GL
data. Overall, it is the most arid site in this analysis (Figure
\ref{fig:site-climate})

\paragraph{Riverton (United States; latitude: 43.0647, longitude:
  -108.4767, elevation: 1699 m)}

Riverton is located in the Rocky Mountains of the United States. It
has a cold desert climate (K\"{o}ppen: BWk). However, due to the
complex terrain other climates exist in close proximity: warm-summer
humid continental climate (K\"{o}ppen: Dfb) to the southwest, and cold
semi-arid climates (K\"{o}ppen: BSk) to the northwest and southeast
\cite{rubel2010}. Twenty-one percent soil moisture observations are
below the wilting point in the CLASS4GL data. Next to Elko and Las
Vegas, it is third most arid site in this analysis (Figure
\ref{fig:site-climate}).

\paragraph{Great Falls (United States; latitude: 47.4614, longitude:
  -111.3847, elevation: 1132 m)}

Great Falls is located just east of the Rocky Mountains in the
northern United Dates. Great Falls has a cold semi-arid climate
(K\"{o}ppen: BSk), but it does receive summer rainfall from isolated
thunderstorms. Less than 1\% of days are below the wilting point in
the CLASS4GL data, and it among the most humid of the arid sites
(Figure \ref{fig:site-climate}).

\section{Results and Discussion}

To reveal the impact of confounding on estimates of the relationship
between soil moisture and evaporation, we compare naive regressions of
evaporation onto soil moisture in our two simulated worlds: (1) the
Realistic World that uses the unaltered CLASS4GL data as forcing for
the CLASS model, and (2) the Deconfounded World where soil moisture's
initial condition is sampled independently from all other
initial/boundary conditions (Section \ref{sec:impact-conf-relat}). We
also explore the degree to which we can adjust/control for this
confounding and estimate the causal impact of soil moisture on
evaporation using passive data from the realistic world (Section
\ref{sec:adjust-conf-can}).

\subsection{The impact of confounding on relationships between soil moisture and evaporation}
\label{sec:impact-conf-relat}

\begin{figure}
  \includegraphics[width=\textwidth]{./sm-causality-figs/reality-deconfounded-comparison.pdf}
  \caption{A comparison of a naive linear model fit in the Realistic
    World to a linear model fit in the Deconfounded World. Uncertainty
    in the ``truth'' (see Section \ref{sec:estim-impact-conf}) is
    represented by error bars.}
\label{fig:reality-deconfounded}
\end{figure}

First, we consider the naive regression of evaporation onto soil
moisture in the Realistic World. In this world, the effect of soil
moisture on evaporation is confounded by the underlying climate state
(Figure \ref{fig:graph}), and the naive regression fit includes bias
due to both confounding and model specification. For the majority of
sites, the naive regression results in biased estimates of the average
effect of soil moisture on evaporation (Figure
\ref{fig:reality-deconfounded}). However, for some sites (Bergen,
Idar-Oberstein, Lindenberg, Milano, and Kelowna), all of which are
among the more humid, even the \textit{sign} of the effect of soil
moisture on evaporation for the naive linear model is opposite of the
truth. That is, the naive approach suggests that evaporation increases
in response to decreasing soil moisture (and increasing
aridity). While there are possible physical explanations for
increasing evaporation with increasing soil aridity (for example,
plants attempting to keep cool in extreme heat\cite{krich2022}), in
this case, those reasons are incorrect: the ``true''average causal
effect, as estimated by the model, is positive for all sites (Figure
\ref{fig:reality-deconfounded}).  For these sites, not accounting for
confounding results in a completely opposite interpretation to the
truth.

Because the naive regression in the Realistic World contains model
specification error in addition to just confounding error, the
Realistic World's naive regression error needs to be interpreted
relative to the Deconfounded World, which has no systematic
confounding error and only model specification error. By comparing
regressions in the Realistic World and the Deconfounded World, we can
gain intuition for the magnitude and sign of confounding error, and
how it varies across sites. If we assume that model specification
error does not change between the Deconfounded and the Realistic
World, which may be a reasonable assumption given that the only
difference between the two worlds are whether soil moisture is sampled
independently (e.g., all other initial and boundary conditions are
sampled with the same co-variate structure), then we can assume that
any difference in regression bias between the Realistic World and the
Deconfounded World is attributable to confounding. If we make this
assumption, which we will re-examine later, then we can see that for
all sites, confounding error biases the estimate of the evaporation
response to soil moisture towards negative values. This bias towards
negative values is consistent with confounders that tend to reduce
soil moisture while enhancing evaporation, or enhance soil moisture
while reducing evaporation. Cloudiness is one such confounder: if
climate conditions that lead to a higher probability of cloudiness
also increase the probability of higher soil moisture, then afternoon
evaporation on higher soil moisture days will be suppressed due to the
decrease in radiation from the clouds, and the naive regression of
evaporation onto soil moisture will be biased towards negative
values. Vapor pressure deficit or atmospheric aridity is another such
confounder: if climate conditions that lead to increases in vapor
pressure deficit (i.e. atmospheric aridity) also increase the
likelihood of low soil moisture conditions, then on days with low soil
moisture afternoon evaporative demand will be enhanced by the
increased vapor pressure deficit, and again, the naive regression will
be biased towards negative values. These are just examples of physical
reasoning used to help explain the negative bias of the evaporation
response to soil moisture due to confounding, but the general point is
that across all sites confounding contributes a negative bias to the
naive estimate of evaporation's response to soil moisture. Approaches
to estimate the effect of soil moisture on evaporation that do not
adjust for confounding will tend to be biased low relative to the true
effect.

Overall, the degree to which confounding shifts regression error
relative to specification error depends on the site. The shift is more
pronounced for the more humid sites (Bergen, Idar-Oberstein,
Lindenberg, Milano, Kelowna, and Quad City). For the more arid sites,
the shift is of a smaller magnitude, and we also observe compensating
errors where the naive approach in the Deconfounded World
(specification error) leads to a positive bias, but, as previously
mentioned, confounding bias shifts the naive approach in the Realistic
World back towards negative values, placing it within the range of the
``true'' response (e.g., Elko and Riverton in Figure
\ref{fig:reality-deconfounded}).

For the more humid sites, and even for some of the more humid of the
arid sites like Spokane and Great Falls, the confounding error is much
larger magnitude relative to the model specification error (Figure
\ref{fig:reality-deconfounded}). This highlights the importance of
accounting for confounding. Even using a very crude linear model to
represent a known non-linear process, the confounding errors are
larger than the model misspecification error. So in some cases,
\textit{identifying and controlling for sources of confounding is more
  important than using a sophisticated non-linear regression
  method}. Of course, ideally we do both: carefully control for
confounding while using a model that captures the relationships in our
data well. For interpretability we are using a linear model.

For Flagstaff both the model specification error and confounding error
are within the range of true values. However for the other arid sites
of Elko, Las Vegas, and Riverton, the interpretation depends on the
degree to which model specification bias is the same between the
Deconfounded and Realistic World. If we believe that specification
bias is exactly the same in both worlds, then it appears that
confounding biases compensate model specification errors: model
specification biases the response positive about the same amount that
confounding biases the result negative. However, the confounding error
(relative to specification error) is still less than for the other,
more humid sites. This matches intuition of the physics: these sites
(Elko, Las Vegas and Riverton) are situated in arid climates (Figure
\ref{fig:site-climate}), with initial soil moisture below the modeled
wilting point on 30\% of days for Elko, 55\% of days for Las Vegas and
20\% of days for Riverton. When soil moisture is below the wilting
point evaporation is approximately zero, independent of all other
variables.  And even when soil moisture is above the wilting point but
still near it, evaporation will be mostly limited by soil moisture and
largely decoupled form the other environmental variables that are
subject to the confounding induced by the underlying climate state. In
plain words, as ecosystems become more water limited, confounding
contributes less bias because soil water limitation is the dominant
determinant of evaporation. In addition to a decrease in confounding
for arid sites, we could also expect an increase in model
specification error: the relationship between evaporation and soil
moisture becomes more nonlinear when ecosystems' soil moisture
straddles the wilting point (Figure \ref{fig:scatter-example}).

When interpreting Figure \ref{fig:reality-deconfounded} we also need
to consider the assumption that model specification error is the same
between both worlds. This is a somewhat reasonable assumption: the
only difference between the two worlds is whether soil moisture is
sampled independently from all of the other variables or not. All
other initial and boundary conditions maintain their same joint
distribution consistent with the underlying climate. However, this
slight difference of soil moisture sampling is still a difference, and
in this case, for the arid sites, the model does appear to fit the
data less well in the Deconfounded World relative to the Realistic
World. Specifically, the data are more non-linear in the Deconfounded
World, and the random error of the model increases more with
increasing soil moisture (in technical terms, the model is more
heteroscedasticity in the Deconfounded World) (Figure
\ref{fig:scatter-example}). Both this increased non-linearity and
increased heteroscedasticity in the Deconfounded World contribute to a
more overt violation of the data modeling assumptions of linear regression,
and likely contribute to a degraded model fit in the Deconfounded
World relative to the realistic world. The reasons for this shift
towards increased non-linearity and heteroscedasticity are related to
the confounding error and how it is suppressed when soil moisture is
below the wilting point.

We established that confounding bias in this soil moisture-evaporation
system is due to covariations that enhance soil moisture while
suppressing evaporation and/or reduce soil moisture while enhancing
evaporation. This means that the low end of the observed soil moisture
spectrum are associated with enhanced conditions for evaporation. When
a site straddles the wilting point (i.e. it is in a strongly water
stressed regime), in the Realistic World these covariations at the low
end of the observed soil moisture spectrum exert no influence over
evaporation; evaporation is independent of the environmental
conditions.  However, when we shift to the Deconfounded world, those
enhanced conditions for evaporation are sampled independently from
soil moisture, so they can occur on days with higher soil moisture
where they dramatically increase evaporation, increasing the nonlinear
difference between evaporation when soil moisture is high and when
soil moisture is at or below the wilting point, and also increasing
the variance of evaporation when soil moisture is high (Figure
\ref{fig:scatter-example}). However, these shifts in non-linearity and
heteroscedasticity between the Realistic and Deconfounded Worlds are
still relatively subtle (Figure \ref{fig:scatter-example}), even if
they are more pronounced for the arid sites. A conservative conclusion
in light of these shifts is that confounding error is more reduced at
the arid sites relative to the more humid sites, reflecting soil
moisture's dominance relative to other environmental factors as a
regulator of evaporation. Additionally, model specification bias of a
linear model is greater at the more arid sites, reflecting the
increased nonlinearity of evaporation dependence on soil moisture for
the more arid sites.

\begin{figure}
  \includegraphics[width=\textwidth]{sm-causality-figs/scatter-las_vegas.pdf}
  \caption{An example of naive ordinary least squares regression fits
    in the Deconfounded World (left panels) and the Realistic World
    (right panels) for Las Vegas. The only difference between the
    Realistic World and the Deconfounded World is that soil moisture
    is sampled independently from the other initial/boundary
    conditions. However, this difference results in a noticeable
    difference of the regression fit, particularly for the more arid
    sites like Las Vegas: in the Deconfounded World the curve is
    slightly more non-linear, and the error is more strongly a
    function of soil moisture, particularly at high soil moisture
    (high heteroscedasticity). This effect is observed most strongly
    in the arid sites. Figures corresponding to all other sites are
    available in Appendix \ref{cha:scatter-distributions}.}
  \label{fig:scatter-example}
\end{figure}

In summary:

\begin{itemize}
\item For all of the humid sites and some of the more mild arid sites
  (Spokane, Great Falls), confounding error is of a larger magnitude
  than model specification, highlighting the importance of adjusting
  for confounding.
\item Confounding error is relatively small for the most arid sites
  where soil moisture limits evaporation, decoupling evaporation from
  the other environmental variables.
\item Linear model specification error increases for the more arid
  sites, where the relationship becomes more nonlinear as soil
  moisture approaches and decreases past the wilting point.
\end{itemize}

\subsection{Adjusting for confounding: can we estimate causal effects?}
\label{sec:adjust-conf-can}

In Section \ref{sec:impact-conf-relat} we examined the errors
introduced by confounding through a comparison of two simulated
worlds: the Realistic World and the Deconfounded World. Here, we
examine how we might calculate causal effects by adjusting and
controlling for confounding in passively observed data. All discussion
in this section will be constrained to the Realistic World (we are
done with the Deconfoundd World). In the Realistic World, we have
estimates of the true interventional effect of varying soil moisture,
so we we can compare the degree to which our estimates of the causal
effect based on passive (i.e. non-experimental) observations is
correct. We can also compare our estimate of the causal effect to the
naive approach where we did not adjust for any sources of confounding.

Following the procedure in Section \ref{sec:attempt-meth-adjust}, we
obtain causal estimates of the evaporation response to soil moisture
using a regression of evaporation onto soil moisture for the k-nearest
neighbors of every day, where neighbor nearness is defined as the
euclidean distance between normalized normalized soil moisture and the
confounders (leaf area index, ground temperature, air temperature,
humidity, cloud cover, wind-speed, boundary layer height (measure of
wind and stability). In other words, days with similar confounders and
soil moisture will be neighbors, and we regress evaporation onto soil
moisture for all the neighboring days to obtain an estimate of the
response of evaporation to soil moisture. We can average this response
over all days to obtain an average response of evaporation to soil
moisture.

\begin{figure}
  \includegraphics[width=\textwidth]{./sm-causality-figs/reality-adjustment-comparison.pdf}
  \caption{A comparison between a naive linear model and a model that
    adjusts for confounding using a k-nearest-neighbors approach
    (Section \ref{sec:attempt-meth-adjust}). Uncertainty in the
    ``truth'' (see Section \ref{sec:estim-impact-conf}) is represented
    by error bars in the top panel. A comparison of errors, defined as
    deviations outside the possible values of truth, is provided in
    the bottom panel.  }
\label{fig:reality-adjustment}
\end{figure}

Adjusting for confounding reduces error/bias substantially for all
sites (Figure \ref{fig:reality-adjustment}). Furthermore, even for the
sites where confounding was a relatively small contributor to bias and
the naive approach yielded a causal estimate within the range of
possible values of the true estimate, using the nearest-neighbor
approach to control for possible confounders did not introduce any new
bias into the causal estimate. So in this case, even when we include
variables that may not be strong sources of confounding, we did not
degrade the quality of our causal estimate. This is important, because
while in our simulated worlds we know the model's initial and boundary
conditions so we know the potential sources of confounding, in the
real world we do not know definitively all the potential sources of
confounding, even if expert knowledge usually provides a strong
suggestion of what they may be. When it comes to confounding, it
appears better to include and adjust for any possible sources of
confounding rather then leaving them out. However, this is still
subject to the caveat that while our simulated world includes many of
the same challenges to causal inference as reality, namely the same
covariation in initial and boundary conditions as the real world,
similar nonlinearities, and feedbacks, the two worlds are not the
same, so there are no guarantees about the degree to which our
conclusions in this simulated world generalize to the realistic
world. Also, even if the simulated world were the same as reality,
each possible causal analysis has different sample sizes and joint
distributions of variables, so one should still check how their model
captures the distribution of the data; for example, whether inclusion
of additional covariates or potential confounders is any different
than just fitting the statistical model to noise.

While our analysis advocates erring on the side of inclusion for
potential sources of confounding, it is very important that one does
not include as a possible source of confounding any variable that
could be affected by the cause. Fortunately, as we saw in Chapter
\ref{cha:caus-infer-proc}, it is easy to exclude such variables by
using the temporal ordering of events: if we never include as
confounders any variables or observations that happen at a time later
than our cause, we avoid the possibility of including cause-affected
variables in our set of confounders. We recommend adjusting for
confounding and considering all possible sources of confounding in the
land atmosphere system, as well as following the guidance in Chapter
\ref{cha:caus-infer-proc}: leverage the temporal ordering of events to
ensure the confounders are not affected by the cause, and check for
the potential impact of the unobserved portion of the state space on
the system.

Our estimates of the causal effect are not always within the range of
possible values of the truth (see Bergen, Idar-Oberstein, Lindenberg,
and Milano in Figure \ref{fig:reality-adjustment}), but they are
close, especially relative to the naive approach. Across site
variability in the evaporation response is also usually greater than
the error of the estimate, and our adjustment technique appears to be
able to differentiate real variability in the causal response of
evaporation to soil moisture across sites. Our method reproduces the
pattern of across-site variability, and our data-driven causal effects
appear useful for studies analyzing and testing hypotheses about how
average evaporation response to soil moisture varies by climate and
ecosystem type.

For the causal effects estimated in this simulated reality, we had the
luxury of interventional experiments for comparison and
validation. However in the real world we have no such luxury, so we
recommend that the practitioner applying these methods to the real
world rigorously ensure that their statistical models capture the
distribution or expectation of the effect (evaporation) conditional on
the cause (soil moisture) and all of variables that must be accounted
for to control for confounding. We wanted to keep the focus on
confounding rather than building statistical models, but we refer the
reader to the bounty of excellent resources on how to build
representative statistical models from data
\cite{shalizi2013,blei2014,gelman1995bayesian}. If the
statistical model can accurately represent the expectation of
evaporation conditional on soil moisture and the confounders, then we
have confidence through the Equation (\ref{eq:6}) that it can also be
used to estimate the causal effect of interest.

Relative to a naive approach, statistically adjusting for confounding
substantially decreases error for most sites, and does not meaningfully
increase error for any sites. Adjusted causal estimates appear to be
capable of differentiating true causal responses across sites (climate
and ecosystem types). We recommend adjusting for confounding wherever
possible.

\section{Conclusions}
\label{sec:conclusions}

In summary, we conclude:

  \begin{itemize}
    \item Confounding bias is largest at the more humid sites, and
      lower at the more arid sites where soil moisture limits
      evaporation and decouples the response from other environmental
      factors (Section \ref{sec:impact-conf-relat}).
  \item At the more humid sites, bias due to confounding is of a
    larger magnitude than model specification bias, even when the
    specified model is a linear model applied to a known non-linear
    process. This highlights the importance of accounting for
    confounding. (Section \ref{sec:impact-conf-relat}).
  \item Statistically adjusting for potential sources of confounding
    improves causal estimates at the highly confounded sites without
    degrading causal estimates at arid, soil moisture-limited sites
    characterized by less confounding bias (Section
    \ref{sec:adjust-conf-can}). There is less confounding bias at
    these sites because soil moisture is a key regulator of
    evaporation in dry conditions.

    \item The estimated causal effects appear to differentiate true
      variations in the causal effects across climates and
      ecosystems. (Section \ref{sec:adjust-conf-can}).
  \end{itemize}

The main point of this paper is that \textbf{confounding can
  substantially bias causal relationships between variables} in a
simulated, but realistic, scenario. We must comprehensively consider
all possible sources of confounding and adjust for them if we want
to estimate true causal effects reliably across climates and
ecosystems.
